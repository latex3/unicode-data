% File load-unicode-punctuation.tex
%
% Copyright 2015 The LaTeX3 Project
%
% It may be distributed and/or modified under the conditions of
% the LaTeX Project Public License (LPPL), either version 1.3c of
% this license or (at your option) any later version. The latest
% version of this license is in the file
% http://www.latex-project.org/lppl.txt.
%
% Issues with this file should be reported at
% https://github.com/latex3/unicode-data
%
% This file parses UnicodeData.txt, provided by the Unicode Consortium,
% and when used with a Unicode-capable engine sets the the \sfcode of
% code points of Unicode classes "Pe" (closing punctuation marks) and "Pf"
% (final quotation marks) to 0 (transparent to TeX).
%
% =============================================================================
%
% The data can only be loaded by Unicode engines. Currently this is limited to
% XeTeX and LuaTeX, both of which define \Umathcode (though that is not used in
% this file).
\ifx\Umathcode\undefined
  \expandafter\endinput
\fi
% This file can be loaded in IniTeX mode so the category codes of |{|, |}| and
% |#| may not be correct. Everything is done in a group so that only the
% settings we want to propagate are made available generally.
\begingroup
  \catcode`\{=1 %
  \catcode`\}=2 %
  \catcode`\#=6 %
% Write some basic information to the log.
  \catcode`\^=7 %
  \newlinechar=`\^^J %
  \message{^^J}%
  \message{load-unicode-punctuation.tex v0.2 (2015-12-03)^^J}%
  \message{Reading Unicode punctuation data^^J}%
% Set up the parser: most of the data from the Unicode file is not
% required, only the code point and character class.
  \def\Pe{Pe}%
  \def\Pf{Pf}%
  \def\parseunicodedataI#1;#2;#3;#4\relax{%
    \def\temp{#3}%
    \ifnum 0\ifx\temp\Pe 1\fi\ifx\temp\Pf 1\fi>0 %
      \global\sfcode"#1=0 %
    \fi
  }%
% From plain: may not be defined (yet).
  \def\loop#1\repeat{\def\body{#1}\iterate}%
  \def\iterate{%
    \body
      \let\next\iterate
    \else
      \let\next\relax
    \fi
    \next
  }%
  \let\repeat\fi
% Actually loading the file requires an input stream, done directly.
% There is a blank line at the end of the data source so there is a check
% here for a |\par|.
  \def\storedpar{\par}%
  \openin0=UnicodeData.txt %
  \loop\unless\ifeof0 %
    \read0 to \unicodedataline  
    \unless\ifx\unicodedataline\storedpar
      \expandafter\parseunicodedataI\unicodedataline\relax
    \fi
  \repeat
  \closein0 %
\endgroup
