% File load-unicode-east-asian-classes.tex
%
% Copyright 2015 The LaTeX3 Project
%
% It may be distributed and/or modified under the conditions of
% the LaTeX Project Public License (LPPL), either version 1.3c of
% this license or (at your option) any later version. The latest
% version of this license is in the file
% http://www.latex-project.org/lppl.txt.
%
% Issues with this file should be reported at
% https://github.com/latex3/unicode-data
%
% This file parses EastAsianWidth.txt and LineBreak.txt, provided by the
% Unicode Consortium, and when used with XeTeX sets \XeTeXcharclass for
% the following classes of code point:
% - "ID" (ideographic)
% - "OP" (opener)
% - "CL" (closer)
% - "NS" (non-starter)
% - "EX" (exclamation)
% - "IS" (infix separator)
% - "CM" (combining marks)
%
% All code points of class "ID" are assigned to a \XeTeXcharclass, but for
% other classes this only occurs when they fall into east Asian width type
% "F", "H" or "W" (full-, half- and wide-width).
%
% As standard, the following mappings between Unicode and XeTeX classes occur
% - "ID" -> 1
% - "OP" -> 2
% - "CL", "NS", "EX", "IS" - > 3
% - "CM" -> 256 (ignored)
% This may be over-ridden by defining \XeTeXcharclass<name> as the numerical
% value to use. For example, to assign Unicode code points of class "ID" to
% \XeTeXcharclass five you would use \chardef\XeTeXcharclassID=5.
%
% This file does _not_ activate XeTeX's inter-character token mechanism
% (`\XeTeXinterchartokenstate` is not set) nor does it install any material in
% the inter-character token registers.
%
% =============================================================================
%
% The data loaded here can currently only be used by XeTeX: check for the
% appropriate primitive.
\ifx\XeTeXcharclass\undefined
  \expandafter\endinput
\fi
% This file can be loaded in IniTeX mode so the category codes of |{|, |}| and
% |#| may not be correct. Everything is done in a group so that only the
% settings we want to propagate are made available generally.
\begingroup
  \catcode`\{=1 %
  \catcode`\}=2 %
% Write some basic information to the log.
  \catcode`\^=7 %
  \newlinechar=`\^^J %
  \message{^^J}%
  \message{load-unicode-east-asian-classes.tex v0.3 (2015-12-03)^^J}%
  \message{Reading Unicode east Asian character class data^^J}%
% A string version of |#| will be needed to look for comment lines in the
% source. Once that is done proper parsing can begin.
  \catcode`\#=12 %
  \def\hash{#}%
  \catcode`\#=6 %
  \def\firsttoken#1#2\relax{#1}%
  \def\parseunicodedataI#1\relax{%
    \unless\if\hash\firsttoken#1?\relax
      \parseunicodedataII#1\relax
    \fi
  }%
% Both files to be parsed here have potential ranges of code points: find the
% first entry and search for the second.
  \def\parseunicodedataII#1;#2 #3\relax{%
    \parseunicodedataIII#1....\relax{#2}%
  }%
% From plain: may not be defined (yet).
  \def\loop#1\repeat{\def\body{#1}\iterate}%
  \def\iterate{%
    \body
      \let\next\iterate
    \else
      \let\next\relax
    \fi
    \next
  }%
  \let\repeat\fi
% For the East Asian width data, save the class of the current token.
  \def\parseunicodedataIII#1..#2..#3\relax#4{%
    \expandafter\def\csname EAW@\number"#1\endcsname{#4}%
    \ifx\relax#2\relax
    \else
      \count0="#1 %
      \loop
        \ifnum\count0<"#2 %
          \advance\count0 by 1 %
          \expandafter\def\csname EAW@\number\count0\endcsname{#4}%
      \repeat
    \fi
  }%
% A shared routine for reading the data files: only one part of the parser
% has to be altered.
  \def\storedpar{\par}%
  \def\readandparse#1{%
    \catcode`\#=12 %
    \everyeof{\noexpand}%
    \edef\temp{\input #1.txt }%
    \message{^^J}%
    \openin0=#1.txt %
% Read two linse from the source file to extract the version information
    \read0 to \unicodedataline
    \message{\unicodedataline ^^J}%
    \read0 to \unicodedataline
    \message{\unicodedataline ^^J}%
    \loop\unless\ifeof0 %
      \read0 to \unicodedataline
      \unless\ifx\unicodedataline\storedpar
        \expandafter\parseunicodedataI\unicodedataline\relax
      \fi
    \repeat
    \catcode`\#=6 %
    \closein0 %
  }%
% Read the east Asian width data: no settings are made at this stage.
  \readandparse{EastAsianWidth}%
% Set up the different line break classes recognised. These might be provided
% before the loader is used, in which case respect those.
  \unless\ifdefined\XeTeXcharclassID
    \chardef\XeTeXcharclassID=1 %
  \fi
  \unless\ifdefined\XeTeXcharclassOP
    \chardef\XeTeXcharclassOP=2 %
  \fi
  \unless\ifdefined\XeTeXcharclassCL
    \chardef\XeTeXcharclassCL=3 %
  \fi
  \unless\ifdefined\XeTeXcharclassEX
    \chardef\XeTeXcharclassEX=3 %
  \fi
  \unless\ifdefined\XeTeXcharclassIS
    \chardef\XeTeXcharclassIS=3 %
  \fi
  \unless\ifdefined\XeTeXcharclassNS
    \chardef\XeTeXcharclassNS=3 %
  \fi
  \unless\ifdefined\XeTeXcharclassCM
    \chardef\XeTeXcharclassCM=256 %
  \fi
  \def\ID{ID}%
% Check the line break class and if necessary the east Asian width for the
% current code point. For code points of class |ID| there may be a range to
% set, and these are always recorded. In other cases check the east Asian width
% and set the class if appropriate.
  \def\parseunicodedataIII#1..#2..#3\relax#4{%
    \def\temp{#4}%
    \ifx\temp\ID
      \ifx\relax#2\relax
        \parseunicodedataIV{#1}{#1}%
      \else
        \parseunicodedataIV{#1}{#2}%
      \fi
    \else
      \ifnum 0%
        \if F\csname EAW@\number"#1\endcsname 1\fi
        \if H\csname EAW@\number"#1\endcsname 1\fi
        \if W\csname EAW@\number"#1\endcsname 1\fi
         >0 %
       \global\XeTeXcharclass"#1=\csname XeTeXcharclass#4\endcsname
      \fi
    \fi
  }%
% As we are inside a loop already, there needs to be a group here to preserve
% the iterator.
  \def\parseunicodedataIV#1#2{%
    \begingroup
      \count0="#1 %
      \loop
        \ifnum\count0<"#2 %
          \global\XeTeXcharclass\count0=1 %
          \advance\count0 by 1 %
      \repeat
    \endgroup
  }%
  \readandparse{LineBreak}%
\endgroup
